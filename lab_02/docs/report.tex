\documentclass{bmstu}

\usepackage{biblatex}
\usepackage{array}
\usepackage{amsmath}

\addbibresource{inc/biblio/sources.bib}

\begin{document}

\makereporttitle
    {Информатика, искусственный интеллект и системы управления} % Название факультета
    {Программное обеспечение ЭВМ и информационные технологии} % Название кафедры
    {лабораторной работе №~2} % Название работы (в дат. падеже)
    {Защита информации} % Название курса (необязательный аргумент)
    {Программная реализация алгоритма DES с применением режима шифрования PCBC} % Тема работы
    {} % Номер варианта (необязательный аргумент)
    {Марченко~В./ИУ7-73Б} % Номер группы/ФИО студента (если авторов несколько, их необходимо разделить запятой)
    {Чиж~И.~С.} % ФИО преподавателя

{\centering \maketableofcontents}

{\centering \chapter*{ВВЕДЕНИЕ}}
\addcontentsline{toc}{chapter}{ВВЕДЕНИЕ}

DES (англ. Data Encryption Standard) --- алгоритм для симметричного шифрования, разработанный фирмой IBM и утвержденный правительством США в 1977 году как официальный стандарт (FIPS 46-3). 
Размер блока для DES равен 64 битам. 
В основе алгоритма лежит сеть Фейстеля с 16 циклами (раундами) и ключом, имеющим длину 56 бит. 
Алгоритм использует комбинацию нелинейных (S-блоки) и линейных (перестановки E, IP, IP$^{-1}$) преобразований~\cite{wiki-des}.

Для DES рекомендовано несколько режимов шифрования~\cite{wiki-des}:
\begin{enumerate}
\item[1)] ECB (англ. electronic code book) --- режим <<электронной кодовой книги>>;
\item[2)] CBC (англ. cipher block chaining) --- режим сцепления блоков;
\item[3)] PCBC (англ. propagating cipher block chaining) --- режим распространяющегося сцепления блоков шифра~\cite{wiki-pcbc};
\item[4)] CFB (англ. cipher feed back) --- режим обратной связи по шифротексту;
\item[5)] OFB (англ. output feed back) --- режим обратной связи по выходу.
\end{enumerate}

Прямым развитием DES в настоящее время является алгоритм Triple DES (3DES). 
В 3DES шифрование/расшифровка выполняются путем троекратного выполнения алгоритма DES~\cite{wiki-des}.

Целью данной лабораторной работы является программная реализация алгоритма шифрования DES с применением режима шифрования PCBC.

Задачи лабораторной работы:
\begin{enumerate}
\item[1)] изучить принцип работы алгоритма DES;
\item[2)] изучить принцип работы режима PCBC;
\item[2)] разработать программное обеспечение для шифрования и расшифровки файлов с применением PCBC;
\item[3)] протестировать разработанное программное обеспечение.
\end{enumerate}

\chapter{Алгоритм шифрования DES}

DES работает с битами (двоичными числами). 
Алгоритм шифрует блоки по 64 бита. 
Для шифрования DES использует ключи, длина которых также составляет 64 бита. 
Однако в алгоритме DES игнорируется каждый восьмой бит ключа, поэтому эффективный размер ключа составляет 56 бит. 
Но в любом случае 64 бита --- это число, вокруг которого организован DES~\cite{grabbe}.

Так как алгоритм шифрует блоки по 64 бита, длина открытого текста должна быть кратна 8 байтам. 
Часто тексты не обладают таким свойством, поэтому в качестве решения данной проблемы при шифровании можно дополнить открытый текст необходимым количеством нулевых байтов~\cite{grabbe}.

DES --- это блочный шифр, то есть он работает с блоками открытого текста заданного размера (64 бита) и возвращает блоки зашифрованного текста того же размера. 
Таким образом, DES приводит к перестановке среди $2^{64}$ возможных комбинаций 64-х бит. 
Каждый блок из 64-х бит делится на два блока по 32 бита каждый, левый полублок $L$ и правый полублок $R$~\cite{grabbe}.

DES работает с 64-битными блоками, используя ключи длиной 56 бит. 
Ключи фактически хранятся в виде последовательности 64-х бит, но каждый восьмой бит ключа не используется (т.~е. биты с порядковыми номерами 8, 16, 24, 32, 40, 48, 56 и 64)~\cite{grabbe}.

Алгоритм DES состоит из двух основных шагов: создание 16-и ключей по 48 бит и непросредственно шифрование блока открытого текста~\cite{grabbe}.

\textbf{Создание 16-и ключей.} 
64-битный ключ уменьшается до 56-битного с помощью таблицы перестановки PC-1 размером $8 \times 7$. 
Т.~е., как было сказано выше, каждый 8-й бит ключа отбрасывается~\cite{grabbe}.

Далее 56-битный ключ разбивается на две части по 28 бит. 
Для получения следующего ключа используется предыдущий. 
В зависимости от раунда шифрования ключ циклически сдвигается на 1 или 2 позиции влево (то есть значения старших битов не теряются, а записываются в младшие). 
Во всех раундах, кроме 1, 2, 9 и 16, сдвиг происходит на две позиции. 
После сдвига в каждом раунде ключи соединяются и с помощью таблицы PC-2 размером $8 \times 6$ уменьшаются до 48-и бит~\cite{grabbe}.

\textbf{Шифрования блока открытого текста.} 
Перед началом процесса шифрования выполняется начальная перестановка с помощью --- таблицы IP размером $8 \times 8$. 
Затем блок делится на левую половину $L_0$ из 32 бит и правую половину $R_0$ такой же длины. 
Затем выполняется 16 итераций с использованием функции $f$, которая работает с двумя блоками --- блоком данных из 32-х бит и ключом $K_n$ из 48-и бит --- для создания блока из 32-х бит. 
Для вычисления $L_0$ и $R_0$ используются следующие формулы~\cite{grabbe}:
\begin{equation}
L_n = R_{n - 1},
\end{equation}
\begin{equation}
R_n = L_{n - 1} \oplus f(R_{n - 1}, K_n).
\end{equation}

Чтобы вычислить $f$, сначала каждый блок $R_{n - 1}$ расширяется с 32-х бит до 48-и. 
Это делается с помощью таблицы E размером $8 \times 6$, которая повторяет некоторые биты в $R_{n - 1}$. 
Затем нужно выполнить операцию XOR для 48-битного блока и 48-битного ключа. 
Теперь есть 8 групп по 6 бит. 
Нкжно использовать их как индексы в таблицах, называемых <<S-блоками>>. 
Каждая группа из шести бит даст индекс в отдельном S-блоке. 
По этому индексу будет находиться 4-битное число. 
Это 4-битное число заменит исходные 6 бит. 
Конечным результатом является то, что восемь групп по 6 бит преобразуются в 8 групп по 4 бита (4-битные выходные данные из S-блоков), всего 32 бита~\cite{grabbe}.

Далее следует перестановка P, которая определена таблицой размером $8 \times 4$. 
P дает 32-битный выход из 32-битного входа путем перестановки бит входного блока~\cite{grabbe}.

С помощью вычисленных $L_n$ и $R_n$ можно перейти к следующему раунду шифрования и вычислению $L_{n + 1}$ и $R_{n + 1}$ по приведенным выше формулам~\cite{grabbe}.

В 16-м раунде нужно объединить оба блока текста $R_{16}L_16$. 
Затем следует финальная переставновка с помощью таблицы IP$^{-1}$ размером $8 \times 8$~\cite{grabbe}.

Таким образом, 64-битный текст открытого текста был зашифрован. 
Данный алгоритм повторяется для всех блоков открытого текста.

Расшифровка --- это просто операция, обратная шифрованию, выполняющая те же шаги, что и при шифровании, но с обратным порядком применения ключей~\cite{grabbe}.

На рисунке~\ref{img:des_encryption} показана схема работы алгоритма DES.

\includeimage
    {des_encryption}
    {f}
    {H}
    {.7\textwidth}
    {Схема работы алгоритма DES~\cite{wiki-des}}

\chapter{Алгоритм режима шифрования PCBC}

Недостатки режима CBC привели к созданию усовершенствованного режима распространяющегося сцепления блоков шифра. 
Естественно, этот режим похож на CBC за исключением того, что предыдущий блок открытого текста и предыдущий блок шифротекста подвергается операции XOR с текущим блоком открытого текста перед шифрованием или после него~\cite{wiki-pcbc}.

Режим шифрования РСВС применяется в протоколе Kerberos 4 версии и позволяет обнаруживать ошибки. 
Данный режим шифрования не является федеральным или международным стандартом. 
Режим РСВС --- вариант режима СВС, обладающий специфическим свойством --- ошибка шифротекста приводит к неправильному расшифрованию всех последующих блоков~\cite{wiki-pcbc}. 

Конечно, этот режим не лишен недостатков. 
Так перестановка двух блоков шифротекста приводит к неправильной расшифровке двух соответствующих блоков открытого текста, но из-за XOR над открытым текстом и шифротекстом дальнейшие ошибки компенсируются. 
Поэтому, если при проверке целостности проверяются только несколько последних блоков расшифрованного текста, можно получить частично испорченное сообщение~\cite{wiki-pcbc}.

На рисунке~\ref{img:pcbc_encryption} показана схема работы режима шифрования PCBC.

\includeimage
    {pcbc_encryption}
    {f}
    {H}
    {1\textwidth}
    {Схема работы режима шифрования PCBC~\cite{wiki-pcbc}}
    
Вектор инициализации --- 64-битная случайно сгенерированная последовательность.

\pagebreak
На рисунке~\ref{img:pcbc_decryption} показана схема работы режима расшифровки PCBC.

\includeimage
    {pcbc_decryption}
    {f}
    {H}
    {1\textwidth}
    {Схема работы режима расшифровки PCBC~\cite{wiki-pcbc}}

\chapter{Требования к входным данным}

Программа принимает три аргумента командной строки. 
Первый аргумент --- путь к файлу, который содержит открытый текст.
Второй аргумент --- путь к файлу, в который будет записан зашифрованный текст. 
Третий аргумент --- путь к файлу, в который будет записан расшифрованный текст.

При наличии ошибок в аргументах командной строки или при передаче на вход программе пустого файла программа выдаст сообщение об ошибке и завершится.

В каталоге cfg есть текстовые файлы с конфигурациями таблиц.

Программное обеспечение для шифрования файлов с помощью алгоритма DES было написано на языке программирования C.

Программа может шифровать любые файлы: .txt, .png, .jpg, .rar и т.~п. 
Максимальный размер файла --- 100 КБ.

\chapter{Тестирование программного обеспечения}

В таблице~\ref{tabular:tests} приведены тесты для проверки корректности работы реализованного программного обеспечения.

\begin{table}[H]
\caption{Тесты}
\label{tabular:tests}
\begin{tabular}{|p{4cm}|p{5cm}|p{6cm}|}
\hline
\textbf{Описание} & \textbf{Открытый текст} & \textbf{Результат шифрования}
\tabularnewline
\hline
Пустой входной файл & & Error: empty input file.
\tabularnewline
\hline
Кол-во аргументов командной строки не равно трем & & Error: program requires 3 filenames.
\tabularnewline
\hline
Один байт & a & a163 75a3 210a 6bf6 (в hex виде)
\tabularnewline
\hline
Обычный открытый текст & Hello world! & 523e 9c00 c0c5 3bd0 e6ca 82c5 6c5e be88 (в hex виде)
\tabularnewline
\hline
\end{tabular}
\end{table}

Помимо приведенных выше тестов были зашифрованы и расшифрованы архивы с текстовыми файлами и фотографиями. 
Все тесты пройдены успешно.

{\centering \chapter*{ЗАКЛЮЧЕНИЕ}}
\addcontentsline{toc}{chapter}{ЗАКЛЮЧЕНИЕ}

В результате выполнения данной лабораторной работы был реализован алгоритм шифрования DES с применением режима PCBC.

Были выполнены следующие задачи:
\begin{enumerate}
\item[1)] изучен принцип работы алгоритма DES;
\item[2)] изучен принцип работы режима PCBC;
\item[2)] разработано программное обеспечение для шифрования и расшифровки файлов с применением PCBC;
\item[3)] протестировано разработанное программное обеспечение.
\end{enumerate}

{\centering \printbibliography[title=СПИСОК ИСПОЛЬЗОВАННЫХ ИСТОЧНИКОВ,heading=bibintoc]}

\end{document}
