\documentclass{bmstu}

\usepackage{biblatex}
\usepackage{array}
\usepackage{amsmath}

\addbibresource{inc/biblio/sources.bib}

\begin{document}

\makereporttitle
    {Информатика, искусственный интеллект и системы управления} % Название факультета
    {Программное обеспечение ЭВМ и информационные технологии} % Название кафедры
    {лабораторной работе №~3} % Название работы (в дат. падеже)
    {Защита информации} % Название курса (необязательный аргумент)
    {Программная реализация алгоритма AES с применением режима шифрования OFB} % Тема работы
    {} % Номер варианта (необязательный аргумент)
    {Марченко~В./ИУ7-73Б} % Номер группы/ФИО студента (если авторов несколько, их необходимо разделить запятой)
    {Чиж~И.~С.} % ФИО преподавателя

{\centering \maketableofcontents}

{\centering \chapter*{ВВЕДЕНИЕ}}
\addcontentsline{toc}{chapter}{ВВЕДЕНИЕ}

AES (англ. Advanced Encryption Standard; также Rijndael) --- симметричный алгоритм блочного шифрования (размер блока 128 бит, ключ 128/192/256 бит), принятый в качестве стандарта шифрования правительством США по результатам конкурса AES. 
Этот алгоритм хорошо проанализирован и сейчас широко используется, как это было с его предшественником DES. 
Национальный институт стандартов и технологий США опубликовал спецификацию AES 26 ноября 2001 года после пятилетнего периода, в ходе которого были созданы и оценены 15 кандидатур. 
26 мая 2002 года AES был объявлен стандартом шифрования. По состоянию на 2009 год AES является одним из самых распространенных алгоритмов симметричного шифрования~\cite{wiki-aes}.

AES можно использовать в совокупности со следующими режимами шифрования:
\begin{enumerate}
\item[1)] ECB (англ. electronic code book) --- режим <<электронной кодовой книги>>;
\item[2)] CBC (англ. cipher block chaining) --- режим сцепления блоков;
\item[3)] PCBC (англ. propagating cipher block chaining) --- режим распространяющегося сцепления блоков шифра~\cite{wiki-ofb};
\item[4)] CFB (англ. cipher feed back) --- режим обратной связи по шифротексту;
\item[5)] OFB (англ. output feed back) --- режим обратной связи по выходу.
\end{enumerate}

Целью данной лабораторной работы является программная реализация алгоритма шифрования AES с применением режима шифрования OFB.

Задачи лабораторной работы:
\begin{enumerate}
\item[1)] изучить принцип работы алгоритма AES;
\item[2)] изучить принцип работы режима OFB;
\item[3)] разработать программное обеспечение для шифрования и расшифрования файлов с применением AES и OFB;
\item[4)] протестировать разработанное программное обеспечение.
\end{enumerate}

\chapter{Алгоритм шифрования AES}

AES является стандартом, основанным на алгоритме Rijndael~\cite{wiki-aes}. 
Он состоит из двух глобальных этапов --- генерации ключей и непосредственно шифрования текста.

Для AES длина P (plaintext) (блока входных данных) и S (state) постоянна и равна 128 бит, а длина шифроключа K (key) составляет 128, 192 или 256 бит. 
При этом исходный алгоритм Rijndael допускает длину ключа и размер блока от 128 до 256 бит с шагом в 32 бита. 
Для обозначения выбранных длин P, S и K в 32-битных словах используется нотация $N_b = 4$ для P и S, $N_k = 4, 6, 8$ для K соответственно для разных длин ключей~\cite{wiki-aes}. 

В начале шифрования P копируется в массив S (каждые последовательные 4 байта P составляют столбец в матрице-массиве S). 
После этого к S применяется процедура AddRoundKey(), и затем S проходит через процедуру трансформации (раунд) 10, 12 или 14 раз (в зависимости от длины ключа), при этом надо учесть, что последний раунд несколько отличается от предыдущих. 
В итоге, после завершения последнего раунда трансформации, S копируется в C (ciphertext)~\cite{wiki-aes}.

Отдельные трансформации SubBytes(), ShiftRows(), MixColumns() и AddRoundKey() --- обрабатывают S~\cite{wiki-aes}. 

Процедура SubBytes() обрабатывает каждый байт S, независимо производя нелинейную замену байтов, используя таблицу замен S-box. 
Такая операция обеспечивает нелинейность алгоритма шифрования~\cite{wiki-aes}. 
При расшифровании C используется обратная таблица замен S-box-inv.

ShiftRows() работает со строками S. 
При этой трансформации строки состояния циклически сдвигаются на $r$ байт по горизонтали в зависимости от номера строки. 
Для нулевой строки $r = 0$, для первой строки $r = 1$ и т.~д. 
Таким образом, каждый столбец S после применения процедуры ShiftRows() состоит из байтов из каждой колонки начального состояния. 
Для алгоритма Rijndael паттерн смещения строк для 128- и 192-битных строк одинаков. Однако для блока размером 256 бит отличается от предыдущих тем, что 2-е, 3-и и 4-е строки смещаются на 1, 3 и 4 байта соответственно. 
Это замечание не относится к AES, так как он использует алгоритм Rijndael только с 128-битными блоками, независимо от размера ключа~\cite{wiki-aes}.

В процедуре MixColumns() четыре байта каждого столбца S смешиваются, используя для этого обратимую линейную трансформацию. 
MixColumns() обрабатывает состояния по столбцам, трактуя каждый из них как полином третьей степени. 
Над этими полиномами производится умножение в поле Галуа. 
Вместе с ShiftRows() MixColumns() вносит диффузию в шифр~\cite{wiki-aes}. 

В процедуре AddRoundKey() K каждого раунда объединяется с S. 
Для каждого раунда K$_{n}$ получается из K$_{n-1}$ c помощью процедуры KeyExpansion(); каждый K такого же размера, что и S. 
Процедура производит побитовый XOR каждого байта S с каждым байтом K~\cite{wiki-aes}. 

На рисунке~\ref{img:aes_encryption} показана схема работы алгоритма AES.

\includeimage
    {aes_encryption}
    {f}
    {H}
    {.7\textwidth}
    {Схема работы алгоритма AES~\cite{wadday}}

\pagebreak
На рисунке~\ref{img:key_schedule} показана схема генерации ключей для раундов шифрования.

\includeimage
    {key_schedule}
    {f}
    {H}
    {.7\textwidth}
    {Схема генерации ключей~\cite{key-schedule}}

\chapter{Алгоритм режима шифрования OFB}

Режим (OFB) обратной связи вывода превращает блочный шифр в синхронный шифр потока: он генерирует ключевые блоки, которые являются результатом сложения с блоками открытого текста, чтобы получить зашифрованный текст. 
Так же, как с другими шифрами потока, зеркальное отражение в зашифрованном тексте производит зеркально отраженный бит в открытом тексте в том же самом местоположении. 
Это свойство позволяет многим кодам с исправлением ошибок функционировать как обычно, даже когда исправление ошибок применено перед кодированием~\cite{wiki-ofb}.

Каждая операция блочного шифра обратной связи вывода зависит от всех предыдущих и поэтому не может быть выполнена параллельно. 
Однако, из-за того, что открытый или зашифрованный текст используются только для конечного сложения, операции блочного шифра могут быть выполнены заранее, позволяя выполнить заключительное шифрование параллельно с открытым текстом. 
Обратная связь по выходу на $k$ разрядов не рекомендуется из соображений криптостойкости~\cite{wiki-ofb}.

Режим OFB имеет следующее преимущество по сравнению с режимом CFB: ошибки, возникающие в результате передачи по каналу с шумом, при дешифровании не <<размазываются>> по всему шифротексту, а локализуются в пределах одного блока. 
Однако открытый текст может быть изменен путем определенных манипуляций с блоками шифротекста. 
Несмотря на то, что OFB-шифрование не поддается распараллеливанию, эффективность процедуры может быть повышена за счет предварительной генерации независимой последовательности блоков~\cite{wiki-ofb}.

Алгоритм расшифрования в режиме OFB полностью совпадает с алгоритмом шифрования. Функция расшифрования блочного алгоритма не используется в данном режиме, т.~к. ключевой поток генерируется только функцией шифрования блока~\cite{wiki-ofb}.

На рисунке~\ref{img:ofb_encryption} показана схема шифрования текста в режиме OFB.

\includeimage
    {ofb_encryption}
    {f}
    {H}
    {1\textwidth}
    {Схема шифрования в режиме OFB~\cite{wiki-ofb}}
    
Вектор инициализации --- 128-битная случайно сгенерированная последовательность.

На рисунке~\ref{img:ofb_decryption} показана схема расшифрования текста в режиме OFB.

\includeimage
    {ofb_decryption}
    {f}
    {H}
    {1\textwidth}
    {Схема расшифрования в режиме OFB~\cite{wiki-ofb}}

\chapter{Требования к входным данным}

Программа принимает два аргумента командной строки.
\begin{enumerate}
\item Первый аргумент --- путь к файлу, который содержит исходный текст (открытый/зашифрованный).
\item Второй аргумент --- путь к файлу, в который будет записан зашифрованный/расшифрованный текст. 
\end{enumerate}

При наличии ошибок в аргументах командной строки или при передаче на вход программе пустого файла программа выдаст сообщение об ошибке и завершится.

В каталоге cfg находятся текстовые файлы с конфигурациями таблиц и значениями ключа и вектора инициализации.

Программное обеспечение для шифрования и расшифрования файлов с помощью алгоритма AES и режима шифрования OFB было написано на языке программирования C.

Программа может зашифровать/расшифровать файлы любых типов.

\chapter{Тестирование программного обеспечения}

В таблице~\ref{tabular:tests} приведены тесты для проверки корректности работы реализованного программного обеспечения.

\begin{table}[H]
\caption{Тесты}
\label{tabular:tests}
\begin{tabular}{|p{4cm}|p{5cm}|p{6cm}|}
\hline
\textbf{Описание} & \textbf{Открытый текст} & \textbf{Результат шифрования}
\tabularnewline
\hline
Пустой входной файл & & Error: input file is empty.
\tabularnewline
\hline
Кол-во аргументов командной строки не равно двум & & Error: program requires 2 parameters.
\tabularnewline
\hline
В файле записан один байт & 1 & 4787 6d68 99b4 a7ff 9b1d 627f 5a65 f913 (в hex виде)
\tabularnewline
\hline
В файле записан ровно один блок (16 байт) & 1234567890123456 & 7587 595b af81 9fc8 ab24 504e 6e56 cf26 (в hex виде)
\tabularnewline
\hline
В файле записано более 16 байт & 1234567890123456789 & 7587 595b af81 9fc8 ab24 504e 6e56 cf26 9558 b32a a4b0 f1af f170 4bc9 f774 6511 (в hex виде)
\tabularnewline
\hline
\end{tabular}
\end{table}

Помимо приведенных выше тестов были зашифрованы и расшифрованы архивы с текстовыми файлами, фотографиями и видео. 
Все тесты пройдены успешно.

{\centering \chapter*{ЗАКЛЮЧЕНИЕ}}
\addcontentsline{toc}{chapter}{ЗАКЛЮЧЕНИЕ}

В результате выполнения данной лабораторной работы был реализован алгоритм шифрования AES с применением режима OFB.

Были выполнены следующие задачи:
\begin{enumerate}
\item[1)] изучен принцип работы алгоритма AES;
\item[2)] изучен принцип работы режима OFB;
\item[2)] разработано программное обеспечение для шифрования и расшифровки файлов с применением AES и OFB;
\item[4)] протестировано разработанное программное обеспечение.
\end{enumerate}

{\centering \printbibliography[title=СПИСОК ИСПОЛЬЗОВАННЫХ ИСТОЧНИКОВ,heading=bibintoc]}

\end{document}
