\documentclass{bmstu}

\usepackage{biblatex}
\usepackage{array}
\usepackage{amsmath}

\addbibresource{inc/biblio/sources.bib}

\begin{document}

\makereporttitle
    {Информатика, искусственный интеллект и системы управления} % Название факультета
    {Программное обеспечение ЭВМ и информационные технологии} % Название кафедры
    {лабораторной работе №~1} % Название работы (в дат. падеже)
    {Защита информации} % Название курса (необязательный аргумент)
    {Программная реализация электронного аналога <<Энигмы} % Тема работы
    {} % Номер варианта (необязательный аргумент)
    {Марченко~В./ИУ7-73Б} % Номер группы/ФИО студента (если авторов несколько, их необходимо разделить запятой)
    {Чиж~И.~С.} % ФИО преподавателя

{\centering \maketableofcontents}

{\centering \chapter*{ВВЕДЕНИЕ}}
\addcontentsline{toc}{chapter}{ВВЕДЕНИЕ}

<<Энигма>> --- переносная шифровальная машина, использовавшаяся для шифрования и расшифрования секретных сообщений. 

Целью данной лабораторной работы является программная реализация алгоритма шифрования, который использовался в шифровальной машине <<Энигма>>.

Задачи лабораторной работы:
\begin{enumerate}
\item[1)] изучить принцип работы шифровальной машины <<Энигма>>;
\item[2)] разработать программное обеспечение для шифрования текста из файла с помощью алгоритма шифрования <<Энигмы>>;
\item[3)] протестировать разработанное программное обеспечение.
\end{enumerate}

\chapter{Шифровальная машина <<Энигма>>}

Первую версию роторной шифровальной машины запатентовал в 1918 году Артур Шербиус~\cite{wiki}.

<<Энигма>> состояла из комбинации механических и электрических систем. 
Механическая часть включала в себя клавиатуру, набор вращающихся дисков --- роторов --- которые были расположены вдоль вала и прилегали к нему, и ступенчатого механизма, двигающего один или несколько роторов при каждом нажатии на клавишу. 
Электрическая часть состояла из электрической схемы, соединяющей между собой клавиатуру, коммутационную панель, лампочки и роторы~\cite{wiki}.

Общий принцип функционирования <<Энигмы>>: при каждом нажатии на клавишу самый правый ротор сдвигается на одну позицию, а при определенных условиях сдвигаются и другие роторы. 
Движение роторов приводит к различным криптографическим преобразованиям при каждом следующем нажатии на клавишу на клавиатуре~\cite{stripp}.

Основные части <<Энигмы>> --- клавиатура, коммутационная панель, три ротора (иногда больше) и рефлектор.

Кабель, помещенный на коммутационную панель, соединял буквы попарно, например, <<E>> и <<Q>> могли быть соединены в пару. 
Эффект состоял в перестановке этих букв до и после прохождения сигнала через роторы. 
Например, когда оператор нажимал <<E>>, сигнал направлялся в <<Q>>, и только после этого уже во входной ротор~\cite{wiki}.

Рефлектор соединял контакты последнего ротора попарно, коммутируя ток через роторы в обратном направлении, но по другому маршруту~\cite{singh}. 
Наличие рефлектора гарантировало, что преобразование, осуществляемое <<Энигмой>>, есть инволюция, то есть расшифрование представляет собой то же самое, что и шифрование~\cite{bauer}. 
Однако наличие рефлектора делает невозможным шифрование какой-либо буквы через саму себя. 
Это было серьезным концептуальным недостатком, впоследствии пригодившимся дешифровщикам~\cite{wiki}.

\chapter{Алгоритм шифрования}

На рисунке~\ref{img:enigma} показан пример шифрования буквы <<Z>>. 
В примере используется коммутационная панель, три ротора типов I, II, и III и рефлектор.  
Все вычисления выполняются в кольце вычетов по модулю 26 (кол-во символов латинского алфавита).

\includeimage
    {enigma}
    {f}
    {H}
    {1\textwidth}
    {Пример шифрования буквы c помощью <<Энигмы>>}
    
У ротора любого типа есть определенная буква, при повороте которой сдвигается соседний левый ротор. 
У некоторых типов роторов таких букв может быть две или даже три. 
Таким образом, каждая буква проходит следующие преобразования: через коммутационную панель, через три ротора, через рефлектор, через три ротора в обратном порядке и еще раз через коммутационную панель. 
Притом правый ротор сдвигается при каждом нажатии на клавишу.

\chapter{Программная реализация}

Требования к входным данным. 
Программа принимает два обязательных и два дополнительных аргумента командной строки. 
Первый дополнительный аргумент --- путь к файлу, который содержит конфигурацию коммутационной панели. 
Второй дополнительный аргумент --- путь к файлу, который содержит конфигурацию рефлектора. 
Конфигурация --- пары больших символов латинского алфавита. 
Каждая пара на следующей строке.  
Например, <<AU>>. 
Тогда <<A>> будет заменяться на <<U>>, а <<U>> --- на <<A>>. 
Третий обязательный аргумент --- путь к файлу, содержащий текст, который нужно зашифровать. 
Четвертый обязательный аргумент --- путь к файлу, в который будет записан зашифрованный текст.

Если первые два параметра не указываются, будет установлена конфигурация коммутационной панели и рефлектора по-умолчанию.

Данная реализация <<Энигмы>> работает только с буквами латинского алфавита. 
На вход можно подавать буквы любого регистра. 
На выходе всегда будет латинская буква в верхнем регистре.

При наличии ошибок в аргументах командной строки или в тексте, который нужно зашифровать, программа выдаст сообщение об ошибке и завершится.

В каталоге config есть три текстовых файла с конфигурациями роторов. 
В первой строке расположены все буквы ротора, во второй --- т. н. <<notch>>, а в третьей --- значение начальной позиции ротора.

Программное обеспечение для шифрования текста с помощью алгоритма <<Энигмы>> было написано на языке программирования C\texttt{+}\texttt{+}.

Программа состоит из точки входа --- функции main --- и классов Steckerbrett (коммутационная панель), Rotor, Reflector и Enigma. 
Класс Enigma является главным в программе. 
Он содержит в себе объект класса Steckerbrett, Reflector и три объекта класса Rotor. 
В листингах~\ref{lst:steckerbrett.h}--\ref{lst:enigma.h} представлены интерфейсы этих классов.

\includelisting{steckerbrett.h}{Интерфейс класса Steckerbrett}

\includelisting{rotor.h}{Интерфейс класса Rotor}

\includelisting{reflector.h}{Интерфейс класса Reflector}

\includelisting{enigma.h}{Интерфейс класса Enigma}

В листинге~\ref{lst:encrypt.cpp} представлена реализация алгоритма шифрования <<Энигмы>>.

\includelisting{encrypt.cpp}{Реализация алгоритма шифрования <<Энигмы>>}

В листинге~\ref{lst:run.sh} показан пример запуска программы.

\includelisting{run.sh}{Пример запуска программы}

В листинге~\ref{lst:example.sh} показан пример работы программы. 
На первой строке показан исходный текст, на второй --- зашифрованный, 
на третьей --- расшифрованный. 
А на четвертой строке показано сообщение, которые является результатом расшифрования при неправильных начальных позициях роторов.

\includelisting{example.sh}{Пример работы программы}

{\centering \chapter*{ЗАКЛЮЧЕНИЕ}}
\addcontentsline{toc}{chapter}{ЗАКЛЮЧЕНИЕ}

В результате выполнения данной лабораторной работы был реализован алгоритм шифрования, который использовался в шифровальной машине <<Энигма>>.

Были выполнены следующие задачи:
\begin{enumerate}
\item[1)] изучен принцип работы шифровальной машины <<Энигма>>;
\item[2)] разработано программное обеспечение для шифрования текста из файла с помощью алгоритма шифрования <<Энигмы>>;
\item[3)] протестировано разработанное программное обеспечение.
\end{enumerate}

{\centering \printbibliography[title=СПИСОК ИСПОЛЬЗОВАННЫХ ИСТОЧНИКОВ,heading=bibintoc]}


\end{document}
