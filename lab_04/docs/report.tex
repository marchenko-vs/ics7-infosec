\documentclass{bmstu}

\usepackage{biblatex}
\usepackage{array}
\usepackage{amsmath}

\addbibresource{inc/biblio/sources.bib}

\begin{document}

\makereporttitle
    {Информатика и системы управления}
    {Программное обеспечение ЭВМ и информационные технологии}
    {лабораторной работе №~4} % Название работы (в дат. падеже)
    {Защита информации} % Название курса (необязательный аргумент)
    {Реализация программы для создания и проверки электронной подписи с использованием алгоритма шифрования RSA и алгоритма хеширования SHA-1} % Тема работы
    {2}
    {Марченко~В./ИУ7-73Б} % Номер группы/ФИО студента (если авторов несколько, их необходимо разделить запятой)
    {Чиж~И.~С.} % ФИО преподавателя

{\centering \maketableofcontents}

{\centering \chapter*{ВВЕДЕНИЕ}}
\addcontentsline{toc}{chapter}{ВВЕДЕНИЕ}

RSA (аббревиатура от фамилий Rivest, Shamir и Adleman) --- криптографический алгоритм с открытым ключом, основывающийся на вычислительной сложности задачи факторизации больших полупростых чисел. 
Криптосистема RSA стала первой системой, пригодной и для шифрования, и для цифровой подписи. 
Алгоритм используется в большом числе криптографических приложений, включая PGP, S/MIME, TLS/SSL, IPSEC/IKE и других~\cite{wiki-rsa}.

Secure Hash Algorithm 1 --- алгоритм криптографического хеширования. 
Описан в RFC 3174. 
Для входного сообщения произвольной длины (максимум $2^{64} - 1$ бит, что примерно равно 2 эксабайта) алгоритм генерирует 160-битное (20 байт) хеш-значение, называемое также дайджестом сообщения, которое обычно отображается как шестнадцатеричное число длиной в 40 цифр. 
Используется во многих криптографических приложениях и протоколах. 
Принципы, положенные в основу SHA-1, аналогичны тем, которые использовались Рональдом Ривестом при проектировании MD4~\cite{wiki-sha}.

Целью данной лабораторной работы является программная реализация алгоритма шифрования RSA и алгоритма хеширования SHA-1.

Задачи лабораторной работы:
\begin{enumerate}
\item[1)] изучить принцип работы алгоритма шифрования RSA;
\item[2)] изучить принцип работы алгоритма хеширования SHA-1;
\item[3)] разработать программное обеспечение для создания и проверки электронной подписи;
\item[4)] протестировать разработанное программное обеспечение.
\end{enumerate}

\chapter{Алгоритм шифрования RSA}

\textbf{Генерация ключей.} 
Для использования RSA нужно сгенерировать два ключа --- открытый (public) и закрытый (private). 
Сначала выбираются два простых числа $p$ и $q$. 
Например, $p = 13$ и $q = 11$. 
Затем вычисляется $n = p \times q = 143$. 
$n$ является второй частю обоих ключей. 
Далее вычисляется $\phi(n) = (p - 1) \times (q - 1) = 120$. 
После этого нужно найти число $e$, которое является взаимно простым с $\phi(n)$. 
Число $e$ должно быть таким, чтобы выполнялось неравенство $1 < e < \phi(n)$. 
Пусть $e = 13$. 
Это первая часть открытого ключа. 
Число $d$ можно найти из формулы $d \times e~mod~\phi(n) = 1$. 
Подойдет значение $d = 37$. 
Таким образом, открытый ключ --- пара $(13,~143)$, а закрытый --- $(37,~143)$.

Замечание: шифровать с помощью ключа можно только такие значения, которые меньше, чем $n$.

\textbf{Шифрование.} 
Допустим, нужно зашифровать число $m = 13$. 
Берем открытый ключ получателя --- $(13,~143)$ --- и вычисляем. 
$c = m^e~mod~n = 13^{13}~mod~143 = 52$.

\textbf{Расшифрование.} 
Берем закрытый ключ получателя --- $(37,~143)$ --- и вычисляем. 
$m = c^d~mod~n = 52^{37}~mod~143 = 13$.

\chapter{Алгоритм хеширования SHA-1}

SHA-1 реализует хеш-функцию, построенную на идее функции сжатия. 
Входами функции сжатия являются блок сообщения длиной 512 бит и выход предыдущего блока сообщения. 
Выход представляет собой значение всех хеш-блоков до этого момента. 
Хеш-значением всего сообщения является выход последнего блока.

Длина входного блока --- 512 бит. 
Последний блок сообщения всегда модифицируется. 
Если его длина меньше 56 байт, в конец сообщения добавляется единичный бит, а далее все заполняется нулями, кроме последних 64-х бит. 
Они являются зарезервированными. 
В них записывается длина исходного сообщения в битах. 
Если последний блок имеют длину 56 или более байт, то добавляется новый блок и заполняется аналогично.

Хеширование осуществляется на протяжении 80-и раундов. 
Для каждых двадцати раундов есть своя функция $f(t;~B,~C,~D)$ и константа $K(t)$ (описаны в RFC 3174).

Сначала блок длиной 512 бит попадает на вход. 
Он делится на 16 слов длиной 32 бита каждое. 
Обозначим эти слова как $W(0),~W(1),~...,~W(15)$. 
Далее с помощью этих 16-и слов вычислим еще 64 слова по правилу: $W(t) = (W(t - 3) \oplus W(t - 8) \oplus W(t - 14) \oplus W(t - 16)) << 1,~t \in [16,~80)$, где $<<$ --- циклический побитовый сдвиг влево, $\oplus$ --- операция XOR.
Каждый раунд обрабатывает одно слово.

Пусть $A = H_0,~B = H_1,~C = H_2,~D = H_3,~E = H_4$ --- текущие значения хеш-слов. 
Значения $H_0,~...,~H_4$ изначально известны (описаны в RFC 3174). 
Затем каждый раунд происходят следующие преобразования: $T = A << 5 + f(t;~B,~C,~D) + E + W(t) + K(t)$, где $z = x + y$ вычисляется как $z = (x + y)~mod~2^{32}$. 
$E = D,~D = C,~C = B << 30,~B = A,~A = T$.

После 80-го раунда вычисляются новые значения хеш-слов: $H_0 = H_0 + A,~H_1 = H_1 + B,~...,~H_4 = H_4 + E$.

Данные шаги выполняются для всех 512-битных блоков $M(i)$ исходного сообщения $M$.

Результат работы алгоритма SHA-1 --- 160-битное значение, которое вычисляется как $H = H_0H_1H_2H_3H_4$.

\chapter{Электронная подпись}

Электронная подпись (ЭП), электронная цифровая подпись (ЭЦП), цифровая подпись (ЦП) позволяет подтвердить авторство электронного документа (будь то реальное лицо или, например, аккаунт в криптовалютной системе). 
Подпись связана как с автором, так и с самим документом с помощью криптографических методов и не может быть подделана с помощью обычного копирования~\cite{wiki-ds}.

ЭЦП --- это реквизит электронного документа, полученный в результате криптографического преобразования информации с использованием закрытого ключа подписи и позволяющий проверить отсутствие искажения информации в электронном документе с момента формирования подписи (целостность), принадлежность подписи владельцу сертификата ключа подписи (авторство), а в случае успешной проверки подтвердить факт подписания электронного документа (неотказуемость)~\cite{wiki-ds}.

Как было сказано в разделе про RSA, с помощью этого алгоритма можно не только шифровать данные, но и создавать электронную подпись. 
Разница в том, что при создании ЭП шифрование происходит с помощью закрытого ключа отправителя, а расшифрование --- с помощью открытого ключа отправителя.

На рисунке~\ref{img:ds-encrypt} показана схема вычисления электронной подписи, а на рисунке~\ref{img:ds-decrypt} --- схема проверки электронной подписи.

\includeimage
    {ds-encrypt}
    {f}
    {H}
    {1\textwidth}
    {Схема вычисления электронной подписи~\cite{hacker}}
    
\includeimage
    {ds-decrypt}
    {f}
    {H}
    {1\textwidth}
    {Схема проверки электронной подписи~\cite{hacker}}

\chapter{Требования к входным данным}

Программа работает в нескольких режимах и принимает разное количество аргументов командной строки. 
Чтобы узнать информацию о режимах и аргументах, можно запустить программу с ключом \textbf{--help} или \textbf{-h}.

Режим генерации RSA ключей: \textbf{--rsa-keys}. 
Простые числа $p$ и $q$ читаются из файлов cfg/p.txt и cfg/q.txt соответственно.

Режим шифрования/расшифрования с помощью алгоритма RSA: \textbf{--rsa exp\_file mod\_file in\_file out\_file}, где exp\_file --- путь к файлу, в котором записана закрытая/открытая экспонента, mod\_file --- путь к файлу, в котором записан модуль, in\_file --- входной файл, out\_file --- выходной файл.

Режим хеширования: \textbf{--sha in\_file out\_file}, где in\_file --- входной файл, out\_file --- выходной файл.

При наличии ошибок в аргументах командной строки программа выдаст сообщение об ошибке и завершится.

Программное обеспечение для создания и проверки электронной подписи было написано на языке программирования C.

Программа работает с файлами любых типов.

\chapter{Тестирование программного обеспечения}

В таблице~\ref{tabular:tests} приведены тесты для проверки корректности работы реализованного программного обеспечения.

\begin{table}[H]
\caption{Тесты}
\label{tabular:tests}
\begin{tabular}{|p{4cm}|p{5cm}|p{6cm}|}
\hline
\textbf{Описание} & \textbf{Открытый текст} & \textbf{Результат хеширования}
\tabularnewline
\hline
Неправильное кол-во аргументов командной строки & & Error: no options were given. 
Use -h or --help for description. parameters.
\tabularnewline
\hline
Пустой входной файл & & da39a3ee 5e6b4b0d 3255bfef 95601890 afd80709
\tabularnewline
\hline
Текст длиной менее 64 байт & Sha & ba79baeb 9f10896a 46ae7471 5271b7f5 86e74640
\tabularnewline
\hline
Текст длиной более 64 байт & В чащах юга жил бы цитрус? Да, но фальшивый экземпляр! & 9e32295f 8225803b b6d5fdfc c0674616 a4413c1b
\tabularnewline
\hline
\end{tabular}
\end{table}
 
Все тесты пройдены успешно.

{\centering \chapter*{ЗАКЛЮЧЕНИЕ}}
\addcontentsline{toc}{chapter}{ЗАКЛЮЧЕНИЕ}

В результате выполнения данной лабораторной работы были реализованы алгоритм шифрования RSA и алгоритм хеширования SHA-1.

Были выполнены следующие задачи:
\begin{enumerate}
\item[1)] изучен принцип работы алгоритма шифрования RSA;
\item[2)] изучен принцип работы алгоритма хеширования SHA-1;
\item[3)] разработано программное обеспечение для создания и проверки электронной подписи;
\item[4)] протестировано разработанное программное обеспечение.
\end{enumerate}

{\centering \printbibliography[title=СПИСОК ИСПОЛЬЗОВАННЫХ ИСТОЧНИКОВ,heading=bibintoc]}

\end{document}
